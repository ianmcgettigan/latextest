%! TeX program = lualatex
\documentclass{scrartcl}

\usepackage{emoji}

\usepackage{amsmath, amssymb}
\usepackage{fancyhdr}

\fancyhead[L]{Accounting for ECO Practice Questions}
\fancyhead[R]{\itshape Fall--Winter, 2024}

\pagestyle{fancy}

\setlength\parindent{0pt} % remove auto indentation
\pagenumbering{gobble} % remove page numbers

\title{Accounting: Practice Questions}
\author{Ian McGettigan}
\date{Fall--Winter, 2024}
\begin{document}

\maketitle

\thispagestyle{empty}

\begin{center}
    The following are some questions that popped into my head while
    studying/learning accounting concepts for the course Accounting for
    ECO. Note the course content may change from year to year.
\end{center}

\bigskip

\textbf{Useful Formulae}

\begin{gather*}
    \text{Assets} = \text{Liabilities} + \text{Equity} \\
    RE_{\text{END} }=RE_{\text{BEG} }+NI-DIV \\
    A = L + PIC + RE \\
    \text{Straight-line annual depreciation expense} =
    \frac{\text{Cost~$-$~Est. Salvage Value} }{\text{Est. Useful life} } \\
    \text{Depreciation expense per unit produced} =
    \frac{\text{Cost~$-$~Est. Salvage Value} }{\text{Est. 
    Total units to be made} } \\
    \text{Double-Declining-Balance} = 
    \text{Double the straight-line depr. rate~$\times$~Asset NBV at 
    beg. year} \\
    \text{Basic EPS} = \frac{\text{Net Income~$-$~Preferred Dividends} }
    {\text{Weighted Avg. Num. Shares Outstanding} }
\end{gather*}


\section{Debits and Credits Basics}

I find debits and credits to be rather confusing, for instinctively when
I think of credit I think increase, and debit meaning decrease. But of
course this is not the case, at least not by design. Fill in the table
below with either ``Increases" or ``Decreases" for each account type.

\begin{center}
\begin{tabular}{|l|l|l|} \hline
    \textbf{Account Type} & \textbf{Debit} & \textbf{Credit} \\ \hline
    Assets & \phantom{Increases} & \phantom{Decreases} \\ \hline 
    Liabilities & & \\ \hline
    Equity (including revenue) & & \\ \hline
    Expenses & & \\ \hline
\end{tabular}
\end{center}
\newpage

\section{Adjusting Entries}

John pays foo inc. \$1,000 in December for services to be provided in
January. Write down the adjusting journal entry for December (year end)
and the adjusting entry in January, once the services have been 
provided. 

\vspace{10em}

The above situation is best referred to as a(n)

a) Accrual

b) Reclassification

c) Balance sheet

d) None of the above

\vspace{10em}

Suppose a company pays for six months of rent in advance, say at the
start of December. Explain why the journal entry for December would 
list this prepaid expense as an asset and not an expense. 
    
\vspace{10em}

Company B seeks legal advice following the suspicion of money laundering.
Company B receives the legal advice in December, but decides to not 
pay for it until January. Suppose the fiscal year ends in December. 
Write down the adjusting journal entry Company B must make in December.

\vspace{10em}

(Credit to ChatGPT). On January 1, a company pays \$24,000 for a 
12-month office rent in advance. The payment covers rent for the entire
year (January to December). 

1. What is the journal entry the company should record on January 1
when it makes the payment?

2. What adjusting entry should the company make at the end of March,
after three months of rent have passed, to recognize the rent expense
for those three months?

\vspace{10em}

\section{Inventory Cost Flow Assumptions}

Under inflationary times, which inventory cost flow assumption will 
result in lower net income: FIFO or LIFO? 

\vspace{10em}

Consider the case of Albert Heijn, a famous grocery store in The 
Netherlands. What inventory cost flow assumption makes most sense for
them? 

\vspace{10em}

\textbf{Phantom Profits}: Suppose Cruiser's inc. sells a boat for 
\$2000 and uses FIFO. The COGS (naturally, taken from the beginning
inventory) is \$1500. But the actual, current cost of replacing the 
boat has, because of inflation, risen to \$1900. Suppose further the
income tax rate is 30\%. Calculate (i) the income tax Cruiser's inc. owes,
(ii) net income, and (iii) the difference between this inflated net
income and what it would be under current costs. 

\vspace{10em}

\section{Noncurrent Assets}

Explain why land is a non-depreciable asset.

\vspace{10em}

Explain why a firm may decide not to capitalize things like office 
supplies, even though they provide many years of future economic
benefits to the firm. Explain clearly which accounting concept is
used in making this judgement. 

\vspace{10em}

In financial accounting, depreciation is an application of the 
\underline{\phantom{matching}} concept.

(a) matching

(b) conservatism

(c) objectivity

(d) none of the above

\vspace{10em}

Why might an asset that is being disposed of have a positive net book
value? List two reasons. 

\vspace{10em}

Suppose equipment that cost \$6000 when new was sold for \$1200, and 
that its estimated salvage value is \$900. Write the journal entry
for what happens. How do the books balance?

\vspace{10em}

Why might a firm acquire another for a price above and beyond the fair
value of the net assets of that other firm (goodwill)?

\vspace{10em}

Suppose Cruiser's inc. purchases a business. The business has the 
following information available: Inventory, \$250000; Land,
\$150000; Buildings, \$400000; Notes payable, \$100000. Determine the
fair value of net assets.

\vspace{10em}

Suppose Cruiser's inc. purchases this business for \$1000000 in cash 
and assuming a note payable of \$100000. Write the journal entry for
the transaction.

\vspace{10em}

Now suppose that goodwill is tested for impairment, and it is found that
the amount in goodwill paid actually exceeds the fair value by 
\$120000. Write the journal entry for the impairment loss adjustment. 

\vspace{10em}

Suppose it was an incorrect impairment adjustment, and really the fair
value was determined to be \$200000. Describe the changes that must be
made to account for this.

\phantom{Goodwill cannot be adjusted upwards! No change}

Why are advertising costs and research and development costs always
treated as expenses?

\vspace{10em}

Machine costs 1800000. Estimated useful life is 8 years. Estimated
salvage value is 180000. Using the double-declining balance method,
find the depreciation expense in the asset's third year. 

\vspace{10em}

\section{Bonds}

Write and explain two reasons why firms may choose to ``call" bonds
back from the bondholders before the scheduled maturity date.

\vspace{10em}

Suppose a firm calls back a bond with face value \$100,000 and book
value of \$95,000 for a total payment of \$102,000. Write the journal
entry effects of this early retirement of the bond.

\vspace{10em}

\section{Shareholder's Equity}

Suppose Racers Inc. sold 40,000 additional shares of its \$2 par value
common stock at a price of \$13 per share. Write the journal entry
effects of this scenario.

\vspace{10em}

Differentiate between \textit{authorized}, \textit{issued}, and 
\textit{outstanding} shares.

\vspace{10em}

Write the journal entry effects for a general transaction of the 
purchase of treasury stock with cash.

\vspace{10em}

\textbf{Preferred Stock Dividend Calculation:}

\medskip

6\%, \$100 par value cumulative preferred stock, 50,000 shares 
authorized, issued, and outstanding. Dividend payable semiannually,
no dividends in arrears. Semiannual preferred dividend amount:

\vspace{5em}

\$4.50, \$75 par value cumulative preferred stock, 50,000 shares 
authorized and issued, 40,000 shares outstanding (10,000 shares of
treasury stock). Dividend payable quarterly, no dividends in arrears.
Quarterly preferred dividend amount:

\vspace{5em}

8\%, \$50 par value cumulative preferred stock, 100,000 shares 
authorized, 60,000 shares issued, 54,000 shares outstanding (6,000
shares of treasury stock). Dividend payable annually. Dividends were
not paid in prior two years. Dividend in current year to pay dividends
in arrears and current year's preferred dividend: 

\vspace{5em}

Explain why many financial managers prefer issuing bonds over preferred
stock. Explain also why investors may prefer bonds to preferred stock
dividends.

\vspace{10em}

Write the journal entry for (i) the declaration date when the board of
directors declares a cash dividend, and (ii) the payment date.

\vspace{10em}

Fill in the blanks. Consider a stock dividend issuance. Common stock
account must \underline{\phantom{increase}} by the number of shares
issued multiplied by the \underline{\phantom{par value}} per share.
Retained earnings \underline{\phantom{decreases}} by the number of 
dividend shares issued multiplied by the \underline{\phantom{market 
value}} per share. Any difference between market price and par value
is recorded in the \underline{\phantom{Additional Paid-In Capital}} 
account. 

\vspace{5em}

Suppose Company XYZ has Common Stock of value \$1600000, 90000 shares
issued at \$20 per share, and 80000 shares outstanding. Net income for
the year is 1 million.

(a) Calculate the amount of capital in excess of par.

\vspace{5em}

(b) Calculate the amount of shares in treasury stock.

\vspace{5em}

(c) Calculate earnings per share (EPS). 

\vspace{10em}

\emoji{hot-pepper}\emoji{hot-pepper}\emoji{hot-pepper}
Based on all the above questions, I have 
asked ChatGPT to make a very difficult question. It is the following:

Company XYZ has issued \$1,000,000 of 6\% semiannual bonds with a 
10-year maturity. The bonds were issued at 97\% of face value.
Midway through the bond term, after five years, interest rates drop,
prompting the company to call the bonds back at a call price of 102\%.
Concurrently, the company decides to issue 50,000 shares of \$5 par 
value common stock at \$30 per share to finance new equipment worth
\$1,500,000 with an estimated useful life of 10 years, no salvage value,
and depreciation calculated using the double-declining balance method. 
Additionally, the company declared and paid a semiannual cumulative 
preferred stock dividend of 8\%, with 20,000 \$100 par value preferred
shares outstanding. There are no dividends in arrears.

\bigskip

\begin{enumerate}
    \item Calculate the total cash paid to call back the bonds and the
        gain or loss from calling the bonds
    \item Write the journal entry for calling the bonds early,
        assuming the book value of the bonds at the time of the call is
        \$960,000.
    \item Write the journal entry for issuing common stock, including
        any effects on the additional paid-in capital account.
    \item Calculate the depreciation expense for the third year of the
        new equipment using the double-declining balance method.
    \item Determine the total preferred dividend that must be paid and 
        journalize the declaration and payment of the preferred
        dividend.
    \item Finally, after the transactions above, calculate the new 
        earnings per share for Company XYZ, given that net income
        for the year is \$2,000,000 and there are 200,000 shares of
        common stock outstanding before the new issuance. 
        Assume no treasury stock. 
\end{enumerate}

\newpage

\section{Income and Cash Flow Statements}

A store's cost for a carpet is \$12 and the owners desire a 20\% gross
profit ratio. Find what is the most likely selling price they will set.

\vspace{10em}

Why might a discount store be satisfied with a relatively low gross profit
ratio, while a boutique seeks a relatively high gross profit ratio to 
achieve profitability?

\vspace{10em}

Explain why a weighted-average number of shares of common stock is used
in calculating (basic) earnings per share.

\vspace{10em}

On September 1, 2022, the beginning of her fiscal year, Cruisers, Inc.
had 200,000 shares of common stock outstanding. On December 3, 2022,
40,000 additional shares were issued for cash. On June 28, 2023,
15,000 shares of common stock were acquired as treasury stock (and are no
longer outstanding). Assume Cruisers Inc. had net income of 
\$1,527,000 for the year ended August 31, 2023, and had 80,000 shares of 
a 7\%, \$50 par value preferred stock outstanding during the year.
Calculate the basic earnings per share of common stock outstanding. 

\vspace{10em}

Consider the indirect method in listing cash flows from operating
activities. Explain why depreciation and amortization expenses are 
added back to net income under this method; why gains/losses are
deducted/added; why accounts receivable/accounts payable are subtracted/%
added.

\vspace{10em}

\section{Extra Review}

Penn company uses a periodic inventory system. 
At the end of the annual accounting period,
the accounting records provided the following information for product 1:

\begin{center}
\begin{tabular}{|lcc|} \hline
     & Units & Unit Cost \\ \hline
    Inventory, prior year & 2,000 & \$5 \\
    For the current year: & & \\ 
    \hspace{1.5em} Purchase, March 21 & 5,000 & \$7 \\ 
    \hspace{1.5em} Purchase, August 1 & 3,000 & \$8 \\
    Inventory, current year & 4,000 & \\ \hline
    
\end{tabular}
\end{center}

Compute ending inventory and cost of goods sold for the current year
under FIFO, LIFO, and average cost inventory costing methods.
(\textbf{Hint:} Set up adjacent columns for each case.)

\vspace{10em}

\end{document}
